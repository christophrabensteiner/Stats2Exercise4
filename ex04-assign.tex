\documentclass[]{article}
\usepackage{lmodern}
\usepackage{amssymb,amsmath}
\usepackage{ifxetex,ifluatex}
\usepackage{fixltx2e} % provides \textsubscript
\ifnum 0\ifxetex 1\fi\ifluatex 1\fi=0 % if pdftex
  \usepackage[T1]{fontenc}
  \usepackage[utf8]{inputenc}
\else % if luatex or xelatex
  \ifxetex
    \usepackage{mathspec}
  \else
    \usepackage{fontspec}
  \fi
  \defaultfontfeatures{Ligatures=TeX,Scale=MatchLowercase}
\fi
% use upquote if available, for straight quotes in verbatim environments
\IfFileExists{upquote.sty}{\usepackage{upquote}}{}
% use microtype if available
\IfFileExists{microtype.sty}{%
\usepackage{microtype}
\UseMicrotypeSet[protrusion]{basicmath} % disable protrusion for tt fonts
}{}
\usepackage[margin=1in]{geometry}
\usepackage{hyperref}
\PassOptionsToPackage{usenames,dvipsnames}{color} % color is loaded by hyperref
\hypersetup{unicode=true,
            pdftitle={Übung 04},
            pdfauthor={Algorithmik und Statistik 2 LAB, SS2019},
            colorlinks=true,
            linkcolor=Maroon,
            citecolor=Blue,
            urlcolor=cyan,
            breaklinks=true}
\urlstyle{same}  % don't use monospace font for urls
\usepackage{color}
\usepackage{fancyvrb}
\newcommand{\VerbBar}{|}
\newcommand{\VERB}{\Verb[commandchars=\\\{\}]}
\DefineVerbatimEnvironment{Highlighting}{Verbatim}{commandchars=\\\{\}}
% Add ',fontsize=\small' for more characters per line
\usepackage{framed}
\definecolor{shadecolor}{RGB}{248,248,248}
\newenvironment{Shaded}{\begin{snugshade}}{\end{snugshade}}
\newcommand{\AlertTok}[1]{\textcolor[rgb]{0.94,0.16,0.16}{#1}}
\newcommand{\AnnotationTok}[1]{\textcolor[rgb]{0.56,0.35,0.01}{\textbf{\textit{#1}}}}
\newcommand{\AttributeTok}[1]{\textcolor[rgb]{0.77,0.63,0.00}{#1}}
\newcommand{\BaseNTok}[1]{\textcolor[rgb]{0.00,0.00,0.81}{#1}}
\newcommand{\BuiltInTok}[1]{#1}
\newcommand{\CharTok}[1]{\textcolor[rgb]{0.31,0.60,0.02}{#1}}
\newcommand{\CommentTok}[1]{\textcolor[rgb]{0.56,0.35,0.01}{\textit{#1}}}
\newcommand{\CommentVarTok}[1]{\textcolor[rgb]{0.56,0.35,0.01}{\textbf{\textit{#1}}}}
\newcommand{\ConstantTok}[1]{\textcolor[rgb]{0.00,0.00,0.00}{#1}}
\newcommand{\ControlFlowTok}[1]{\textcolor[rgb]{0.13,0.29,0.53}{\textbf{#1}}}
\newcommand{\DataTypeTok}[1]{\textcolor[rgb]{0.13,0.29,0.53}{#1}}
\newcommand{\DecValTok}[1]{\textcolor[rgb]{0.00,0.00,0.81}{#1}}
\newcommand{\DocumentationTok}[1]{\textcolor[rgb]{0.56,0.35,0.01}{\textbf{\textit{#1}}}}
\newcommand{\ErrorTok}[1]{\textcolor[rgb]{0.64,0.00,0.00}{\textbf{#1}}}
\newcommand{\ExtensionTok}[1]{#1}
\newcommand{\FloatTok}[1]{\textcolor[rgb]{0.00,0.00,0.81}{#1}}
\newcommand{\FunctionTok}[1]{\textcolor[rgb]{0.00,0.00,0.00}{#1}}
\newcommand{\ImportTok}[1]{#1}
\newcommand{\InformationTok}[1]{\textcolor[rgb]{0.56,0.35,0.01}{\textbf{\textit{#1}}}}
\newcommand{\KeywordTok}[1]{\textcolor[rgb]{0.13,0.29,0.53}{\textbf{#1}}}
\newcommand{\NormalTok}[1]{#1}
\newcommand{\OperatorTok}[1]{\textcolor[rgb]{0.81,0.36,0.00}{\textbf{#1}}}
\newcommand{\OtherTok}[1]{\textcolor[rgb]{0.56,0.35,0.01}{#1}}
\newcommand{\PreprocessorTok}[1]{\textcolor[rgb]{0.56,0.35,0.01}{\textit{#1}}}
\newcommand{\RegionMarkerTok}[1]{#1}
\newcommand{\SpecialCharTok}[1]{\textcolor[rgb]{0.00,0.00,0.00}{#1}}
\newcommand{\SpecialStringTok}[1]{\textcolor[rgb]{0.31,0.60,0.02}{#1}}
\newcommand{\StringTok}[1]{\textcolor[rgb]{0.31,0.60,0.02}{#1}}
\newcommand{\VariableTok}[1]{\textcolor[rgb]{0.00,0.00,0.00}{#1}}
\newcommand{\VerbatimStringTok}[1]{\textcolor[rgb]{0.31,0.60,0.02}{#1}}
\newcommand{\WarningTok}[1]{\textcolor[rgb]{0.56,0.35,0.01}{\textbf{\textit{#1}}}}
\usepackage{longtable,booktabs}
\usepackage{graphicx,grffile}
\makeatletter
\def\maxwidth{\ifdim\Gin@nat@width>\linewidth\linewidth\else\Gin@nat@width\fi}
\def\maxheight{\ifdim\Gin@nat@height>\textheight\textheight\else\Gin@nat@height\fi}
\makeatother
% Scale images if necessary, so that they will not overflow the page
% margins by default, and it is still possible to overwrite the defaults
% using explicit options in \includegraphics[width, height, ...]{}
\setkeys{Gin}{width=\maxwidth,height=\maxheight,keepaspectratio}
\IfFileExists{parskip.sty}{%
\usepackage{parskip}
}{% else
\setlength{\parindent}{0pt}
\setlength{\parskip}{6pt plus 2pt minus 1pt}
}
\setlength{\emergencystretch}{3em}  % prevent overfull lines
\providecommand{\tightlist}{%
  \setlength{\itemsep}{0pt}\setlength{\parskip}{0pt}}
\setcounter{secnumdepth}{0}
% Redefines (sub)paragraphs to behave more like sections
\ifx\paragraph\undefined\else
\let\oldparagraph\paragraph
\renewcommand{\paragraph}[1]{\oldparagraph{#1}\mbox{}}
\fi
\ifx\subparagraph\undefined\else
\let\oldsubparagraph\subparagraph
\renewcommand{\subparagraph}[1]{\oldsubparagraph{#1}\mbox{}}
\fi

%%% Use protect on footnotes to avoid problems with footnotes in titles
\let\rmarkdownfootnote\footnote%
\def\footnote{\protect\rmarkdownfootnote}

%%% Change title format to be more compact
\usepackage{titling}

% Create subtitle command for use in maketitle
\providecommand{\subtitle}[1]{
  \posttitle{
    \begin{center}\large#1\end{center}
    }
}

\setlength{\droptitle}{-2em}

  \title{Übung 04}
    \pretitle{\vspace{\droptitle}\centering\huge}
  \posttitle{\par}
    \author{Algorithmik und Statistik 2 LAB, SS2019}
    \preauthor{\centering\large\emph}
  \postauthor{\par}
      \predate{\centering\large\emph}
  \postdate{\par}
    \date{Bis: Sonntag, 23. Juni 2019, 23:59 Uhr}


\begin{document}
\maketitle

Bitte um Beachtung der
\href{https://weblearn.fh-kufstein.ac.at/mod/page/view.php?id=46374}{Übungs-Policy}
für genaue Anweisungen und einige Beurteilungsnotizen. Fehler bei der
Einhaltung ergeben Punktabzug.

\hypertarget{aufgabe-1}{%
\subsection{Aufgabe 1}\label{aufgabe-1}}

\textbf{{[}10 points{]}} Für diese Frage verwenden wir die
\texttt{OJ}Daten aus dem \texttt{ISLR}-Paket. Wir werden versuchen, die
Variable ``Purchase'' vorherzusagen. Nachdem Sie \texttt{uin} zu Ihrem
\texttt{UIN} geändert haben, verwenden Sie den folgenden Code, um die
Daten aufzuteilen.

\begin{Shaded}
\begin{Highlighting}[]
\KeywordTok{library}\NormalTok{(ISLR)}
\KeywordTok{library}\NormalTok{(caret)}
\NormalTok{uin =}\StringTok{ }\DecValTok{123456789}
\KeywordTok{set.seed}\NormalTok{(uin)}
\NormalTok{oj_idx =}\StringTok{ }\KeywordTok{createDataPartition}\NormalTok{(OJ}\OperatorTok{$}\NormalTok{Purchase, }\DataTypeTok{p =} \FloatTok{0.5}\NormalTok{, }\DataTypeTok{list =} \OtherTok{FALSE}\NormalTok{)}
\NormalTok{oj_trn =}\StringTok{ }\NormalTok{OJ[oj_idx,]}
\NormalTok{oj_tst =}\StringTok{ }\NormalTok{OJ[}\OperatorTok{-}\NormalTok{oj_idx,]}
\end{Highlighting}
\end{Shaded}

\textbf{(a)} Stimmen Sie ein SVM mit linearem Kernel mit 5-facher
Cross-Validierung auf die Trainingsdaten ab. Verwenden Sie das folgende
Wertgitter für \texttt{C}. Berichten Sie die gewählten Werte aller
Tuningparameter + Testgenauigkeit.

\begin{Shaded}
\begin{Highlighting}[]
\NormalTok{lin_grid =}\StringTok{ }\KeywordTok{expand.grid}\NormalTok{(}\DataTypeTok{C =} \KeywordTok{c}\NormalTok{(}\DecValTok{2} \OperatorTok{^}\StringTok{ }\NormalTok{(}\OperatorTok{-}\DecValTok{5}\OperatorTok{:}\DecValTok{5}\NormalTok{)))}
\end{Highlighting}
\end{Shaded}

\textbf{(b)} Abstimmung eines SVM mit Polynomkern auf die Trainingsdaten
mittels 5-facher Cross-Validierung. Geben Sie kein Tuning-Grid an.
(\texttt{caret} wird einen für Sie erstellen.) Berichten Sie die
gewählten Werte aller Tuning-Parameter. Berichten Sie über die
Genauigkeit der Testdaten.

\textbf{(c)} Stimmen Sie ein SVM mit Radialkernel mit 5-facher
Cross-Validierung auf die Trainingsdaten ab. Verwenden Sie das folgende
Wertgitter für \texttt{C} und \texttt{sigma}. Berichten Sie die
gewählten Werte aller Tuningparameter. Berichten Sie über die
Genauigkeit der Testdaten.

\begin{Shaded}
\begin{Highlighting}[]
\NormalTok{rad_grid =}\StringTok{ }\KeywordTok{expand.grid}\NormalTok{(}\DataTypeTok{C =} \KeywordTok{c}\NormalTok{(}\DecValTok{2} \OperatorTok{^}\StringTok{ }\NormalTok{(}\OperatorTok{-}\DecValTok{2}\OperatorTok{:}\DecValTok{3}\NormalTok{)), }\DataTypeTok{sigma  =} \KeywordTok{c}\NormalTok{(}\DecValTok{2} \OperatorTok{^}\StringTok{ }\NormalTok{(}\OperatorTok{-}\DecValTok{3}\OperatorTok{:}\DecValTok{1}\NormalTok{)))}
\end{Highlighting}
\end{Shaded}

\textbf{(d)} Stimmen Sie einen Random Forest mit einer 5-fachen
Kreuzvalidierung ab. Berichten Sie die gewählten Werte aller
Tuningparameter. Berichten Sie über die Genauigkeit der Testdaten.

\textbf{(e)} Fassen Sie die obigen Genauigkeiten zusammen. Welche
Methode hat am besten funktioniert? Warum?

\hypertarget{aufgabe-2}{%
\section{Aufgabe 2}\label{aufgabe-2}}

\textbf{{[}10 points{]}} Verwenden Sie für diese Frage die Daten in
\texttt{clust\_data.csv}. Wir werden versuchen, diese Daten mit
\(k\)-means zu bündeln. Aber, welche \(k\) sollen wir verwenden?

\textbf{(a)} Wenden Sie \(k\)-means 15 mal auf diese Daten an, wobei Sie
die Anzahl der Zentren von 1 bis 15 verwenden. Verwenden Sie jedes Mal
\texttt{nstart\ =\ 10} und speichern Sie den Wert \texttt{tot.withinss}
aus dem resultierenden Objekt. (Hinweis: Schreiben Sie eine
for-Schleife.) Die \texttt{tot.withinss} misst, wie variabel die
Beobachtungen innerhalb eines Clusters sind, das wir gerne niedrig
halten würden. Offensichtlich wird dieser Wert also mit mehr Zentren
niedriger sein, egal wie viele Cluster es wirklich gibt. Zeichne diesen
Wert gegen die Anzahl der Zentren auf. Suchen Sie nach einem
``Ellenbogen'', der Anzahl der Zentren, in denen die Verbesserung
plötzlich wegfällt. Basierend auf dieser Darstellung, wie viele Cluster
sollten Ihrer Meinung nach für diese Daten verwendet werden?

\textbf{(b)} Wenden Sie \(k\)-means für die von Ihnen gewählte Anzahl
von Zentren erneut an. Wie viele Beobachtungen werden in jedem Cluster
platziert? Was ist der Wert von \texttt{tot.withinss}?

\textbf{(c)} Visualisieren Sie diese Daten. Plotten Sie die Daten mit
den ersten beiden Variablen und färben Sie die Punkte entsprechend des
\(k\)-means clusterings. Basierend auf diesem Plot, denken Sie, dass Sie
eine gute Wahl für die Anzahl der Zentren getroffen haben? (Kurze
Erklärung.)

\textbf{(d)} Verwenden Sie PCA, um diese Daten zu visualisieren. Plotten
Sie die Daten mit den ersten beiden Hauptkomponenten und färben Sie die
Punkte entsprechend dem \(k\)-means Clustering. Basierend auf diesem
Plot, denken Sie, dass Sie eine gute Wahl für die Anzahl der Zentren
getroffen haben? (Kurze Erklärung.)

\textbf{(e)} Berechnen Sie den Anteil der Variation, der durch die
Hauptkomponenten erklärt wird. Machen Sie eine Darstellung des
kumulierten Anteils erklärt. Wie viele Hauptkomponenten sind notwendig,
um 95\% der Variation der Daten zu erklären?

\hypertarget{aufgabe-3}{%
\section{Aufgabe 3}\label{aufgabe-3}}

\textbf{{[}10 points{]}} Für diese Frage werden wir auf die
\texttt{USArrests} Daten aus den Notizen zurückkommen. (Dies ist ein
Standarddatensatz von \texttt{R}.)

\textbf{(a)} Führen Sie hierarchisches Clustering sechsmal durch.
Berücksichtigen Sie alle möglichen Kombinationen von Verknüpfungen
(Average, Single, Complete) und Datenskalierung. (Skaliert, Nicht
skaliert.)

\begin{longtable}[]{@{}ll@{}}
\toprule
Linkage & Scaling\tabularnewline
\midrule
\endhead
Single & No\tabularnewline
Average & No\tabularnewline
Complete & No\tabularnewline
Single & Yes\tabularnewline
Average & Yes\tabularnewline
Complete & Yes\tabularnewline
\bottomrule
\end{longtable}

Schneiden Sie das Dendrogramm jedes Mal auf eine Höhe, die zu vier
verschiedenen Clustern führt. Plotten Sie die Ergebnisse mit einer Farbe
für jeden Cluster.

\textbf{(b)} Basierend auf den obigen Plots, erscheint eines der
Ergebnisse nützlicher als die anderen? (Es gibt hier keine richtige
Antwort.) Wählen Sie Ihren Favoriten. (Nochmals, keine richtige
Antwort.)

\textbf{(c)} Verwenden Sie die Dokumentation zu \texttt{?hclust}, um
weitere mögliche Verknüpfungen zu finden. Such dir einen aus und
probiere ihn aus. Vergleichen Sie die Ergebnisse mit Ihren Favoriten von
\textbf{(b)}. Ist es anders?

\textbf{(d)} Verwenden Sie die Dokumentation zu \texttt{?dist}, um
andere mögliche Entfernungsmessungen zu finden. (Wir haben
\texttt{euklidisch} verwendet.) Wählen Sie eine (nicht \texttt{binär})
und versuchen Sie es. Vergleichen Sie die Ergebnisse mit Ihren Favoriten
von \textbf{(b)}. Ist es anders?


\end{document}
